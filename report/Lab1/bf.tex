\documentclass[UTF8]{ctexart}

%----------- 版式与字体 ----------------
\usepackage{geometry}
\geometry{a4paper,margin=2.5cm}
\setlength{\parindent}{2em}
\setlength{\parskip}{0.5em}
\renewcommand{\baselinestretch}{1.3}

%----------- 列表与颜色 ----------------
\usepackage[shortlabels]{enumitem}
\setlist[enumerate]{label=\arabic*.}
\usepackage{xcolor}
\usepackage{titlesec}

%----------- 数学宏包 ------------------
\usepackage{amsmath}
\usepackage{amsfonts}
\usepackage{amssymb}

%----------- 题注样式(节标题) ---------
\titleformat{\section}
  {\zihao{4}\bfseries}{\chinese{section}、}{0.5em}{}
\titleformat{\subsection}
  {\zihao{5}\bfseries}{\arabic{section}.\arabic{subsection}}{0.5em}{}

%----------- 正文开始 ------------------
\begin{document}

\begin{center}
  {\zihao{3}\bfseries 人工智能实验报告}\\[1ex]
  {\zihao{4} 实验一\quad 知识表示、推理与搜索}
\end{center}

% ----------- 信息栏 -------------------
\renewcommand{\arraystretch}{1.6}
\begin{tabular}{p{5cm}p{5.5cm}p{3cm}p{5.5cm}}
  学院:计算机与通信工程 & 专业:计算机科学与技术 & 班级:计221\\
  姓名:乔彦博 & 学号:U202242223 & 日期:2025.4.27\\
\end{tabular}

% ----------- 一、实验目标 --------------
\section*{实验目标}
\begin{enumerate}
  \item 熟练掌握知识表示的多种基本方法,包括状态空间法、产生式系统等,并能根据不同问题灵活选择和运用。
  \item 准确实现经典搜索算法,如广度优先搜索(\textbf{BFS})、A\textsuperscript{*}算法(启发式函数)等,理解其算法原理和执行过程。
  \item \textbf{深入分析不同搜索策略在效率方面的差异,通过实验结果进行可视化、对比和评估。}
\end{enumerate}

% ----------- 二、实验内容 --------------
\section*{实验内容(根据实验要求文档)}
\begin{enumerate}
%--------------- 1 ---------------------
  \item 问题描述:
  \begin{itemize}
    \item \textbf{八数码问题}:将 $3\times3$ 棋盘视为 9 个位置的状态空间,每个状态用矩阵形式 $((1,2,3),(4,0,6),(7,5,8))$ 表示,其中 0 代表空格。空格可与上下左右的数字交换位置,从而产生新状态。
    \item \textbf{传教士与野人问题}:用三元组 $(m_{\text{left}},w_{\text{left}},\text{boat\_side})$ 表示左岸传教士人数、野人人数及船的位置(0 表示在右岸,1 表示在左岸)。遵循以下约束:
      \begin{itemize}
        \item 船一次可载 1–2 人;
        \item 任一岸上若有传教士,其数量必须不少于野人数量。
      \end{itemize}
    \item \textbf{搜索算法实现要求}:
      \begin{itemize}
        \item \textbf{广度优先搜索(BFS)}:保证找到最短路径解,需维护已访问状态集合,记录扩展节点数与解路径长度;
        \item \textbf{A* 算法}:在 BFS 基础上集成启发式函数,八数码问题使用曼哈顿距离,传教士与野人问题使用左岸总人数作为估计值;
        \item 对比分析两种算法在节点扩展量、运行时间上的差异,评估启发式函数对搜索效率的影响。
      \end{itemize}
  \end{itemize}

%--------------- 2 ---------------------
  \item \textbf{实现工具:}
  \begin{itemize}
    \item \textbf{编程语言}\quad Python 3.11.12(macOS Apple Silicon 原生支持,单线程即可满足实验规模);
    \item \textbf{主要库依赖}
      \begin{itemize}
        \item \texttt{collections.deque}:O(1) 双端队列,用于 BFS;
        \item \texttt{heapq}:二叉最小堆,实现 A\textsuperscript{*} 优先队列;
        \item \texttt{time}:高精度计时(\texttt{perf\_counter()});
        \item \texttt{numpy}:矩阵操作(八数码可选);
        \item \texttt{matplotlib}:性能结果可视化。
      \end{itemize}
    \item \textbf{开发环境}\quad VS Code +\;Jupyter Notebook,启用 \texttt{-Xfrozen\_modules} 以缩短启动时间。
  \end{itemize}

%--------------- 3 ---------------------
  \item \textbf{实现方案:}
  \begin{enumerate}[label=(\alph*)]
    \item \textbf{状态设计}
      \begin{itemize}
        \item 八数码:定长元组 \verb|Board = Tuple[Tuple[int,...],...]|,并预生成 \verb|GOAL_POS| 以 O(1) 计算曼哈顿距离;
        \item 传教士与野人:三元组 \verb|State = (ml, cl, boat)|,总合法状态 $\le 32$,便于调试。
      \end{itemize}
    \item \textbf{搜索骨架}
      \begin{itemize}
        \item 公共父类 \verb|Node| 保存 \verb|state|, \verb|g|, \verb|parent|,使用 \verb|dataclass(slots=True)|;
        \item BFS 用 \verb|deque|,层次展开;A\textsuperscript{*} 用 \verb|heapq|,按 \(f=g+h\) 取最小。
      \end{itemize}
    \item \textbf{启发式函数 \(h\)}
      \begin{itemize}
        \item 传教士与野人:\(h = ml + cl + \mathbf{1}_{\text{boat在右}}\);
        \item 传教士与野人:\(h = ml + cl + \mathbf{1}_{\text{boat在右}}\)。
      \end{itemize}
    \item \textbf{复杂度与边界条件}
      \begin{itemize}
        \item 八数码判重基于棋盘哈希;传教士问题每次生成最多 5 个后继并即时过滤非法状态;
        \item 若输入即为目标,算法 O(1) 返回;若无解(如八数码奇偶错位)抛出 \verb|RuntimeError|。
      \end{itemize}
  \end{enumerate}

%--------------- 4 ---------------------
  \item \textbf{实现内容与实验结果:}
  \begin{enumerate}[label=(\alph*)]
    \item \textbf{核心代码结构}
      \begin{itemize}
        \item \verb|eight_puzzle_search.py|:实现 BFS 与 A\textsuperscript{*},附 \verb|demo()|;
        \item \verb|missionaries_cannibals_search.py|:同上;
        \item 统一接口:\verb|bfs(start)| / \verb|a_star(start)| → \verb|(path, expanded, elapsed)|。
      \end{itemize}
    \item \textbf{实验流程}
      \begin{enumerate}[label=\roman*.]
        \item 设定初始状态;
        \item 分别调用 BFS 与 A\textsuperscript{*};
        \item 记录路径长度、扩展节点数、耗时;
        \item 可多次测量取均值;
        \item 用 \texttt{matplotlib} 绘制条形图比较。
      \end{enumerate}
    \item \textbf{实验结果摘要}
      \begin{center}
        \begin{tabular}{lcccc}
          \hline
          \textbf{问题} & \textbf{算法} & \textbf{路径长度} & \textbf{扩展节点} & \textbf{运行时间/s} \\
          \hline
          八数码 & BFS & 2 & 2 & 0.0000 \\
          八数码 & A$^{*}$ & 2 & 2 & 0.0000 \\
          传教士与野人 & BFS & 11 & 13 & $5.1\times10^{-5}$ \\
          传教士与野人 & A$^{*}$ & 11 & 14 & $4.4\times10^{-5}$ \\
          \hline
        \end{tabular}
      \end{center}
\vspace{0.5em}
\noindent
\textbf{可视化示例代码节选}:
\begin{verbatim}
import matplotlib.pyplot as plt
labels = ['8-Puzzle', 'Missionaries']
bfs_nodes = [2, 13]
astar_nodes = [2, 14]
x = range(len(labels))
plt.bar(x, bfs_nodes, label='BFS')
plt.bar(x, astar_nodes, bottom=bfs_nodes, label='A*')
plt.ylabel('Expanded Nodes')
plt.xticks(x, labels)
plt.legend(); plt.show()
\end{verbatim}
  \end{enumerate}
\end{enumerate}

% ---------------- 实验总结 ----------------
\section*{实验总结}

\textbf{1.实验结论}  
BFS 作为无启发盲目搜索,在浅层或状态空间极小问题上已能快速找到最优解;但随着深度增长,其节点爆炸现象明显。  
A$^{*}$ 依赖启发式函数质量:  
\begin{itemize}
  \item 对八数码,简单曼哈顿距即可显著剪枝(在更深乱序样例中扩展节点可减少数十倍);  
  \item 对传教士与野人,由于状态空间仅 32 个结点,弱启发式并未体现优势,甚至出现 1 个结点的轻微反超。
\end{itemize}

\textbf{2.启发式函数的作用}  
曼哈顿距满足可采纳(不高估)与一致性($h(n) \leq c(n,n') + h(n')$),保证 A$^{*}$ 找到最优解且不重复扩展。  
若以 Pattern Data\-base 等更强启发式替换,可继续降低时间/空间复杂度,体现“\emph{领域知识 $\Rightarrow$ 搜索效率}”这一核心思想。

\textbf{3.存在问题与改进方向}  
\begin{enumerate}[label=(\alph*)]
  \item \textbf{随机化难例}:八数码随机打乱 ≥ 20 步,能更直观展示 A$^{*}$ 优势;  
  \item \textbf{启发式优化}:传教士问题可改用 $\lceil (ml+cl)/2 \rceil$ 作为最少渡河轮次下界;  
  \item \textbf{算法拓展}:实现 IDS(深度迭代加深)或双向 A$^{*}$,在有向无障碍图中可进一步加速;  
  \item \textbf{性能评估}:加入峰值队列长度及内存 Profiling,获得更全面的资源消耗曲线。
\end{enumerate}

\textbf{4.收获与体会}  
通过本实验深刻体会到:\emph{“搜索策略选择 + 启发式设计”} 才是 AI 问题求解效率的关键。掌握统一的搜索框架后,投入时间打磨启发式往往收益最高。此外,严谨的实验统计与可视化可帮助我们快速定位瓶颈、迭代算法。
)

\end{document}
